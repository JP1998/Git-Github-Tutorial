\documentclass{beamer}
\usetheme{Montpellier}
\usepackage{german}

\usepackage[utf8x]{inputenc}

\usepackage{amssymb}
\usepackage{amsmath}

\usepackage{braket}

\usepackage{listings}
\usepackage{color}

\definecolor{sourcecodered}{rgb}{0.6,0,0} % for strings
\definecolor{sourcecodegreen}{rgb}{0.25,0.5,0.35} % comments
\definecolor{sourcecodepurple}{rgb}{0.5,0,0.35} % keywords
\definecolor{sourcecodedocblue}{rgb}{0.25,0.35,0.75} % javadoc

\lstset{
	language=C,                                    % the language of the code
	basicstyle=\ttfamily\small,                    % the size of the fonts that are used for the code
	keywordstyle=\ttfamily\small,                  % keyword style
	stringstyle=\ttfamily\small,                   % string literal style
	commentstyle=\ttfamily\small,                  % comment style
	numbers=none,                                  % where to put the line-numbers; possible values are (none, left, right)
	numberstyle=\small\color{black},               % the style that is used for the line-numbers
	numbersep=6pt,                                 % how far the line-numbers are from the code
	tabsize=4,	                   	               % sets default tabsize to 4 spaces
	showspaces=false,                              % show spaces everywhere adding particular underscores; it overrides 'showstringspaces'
	showstringspaces=false,                        % underline spaces within strings only
	showtabs=false,                                % show tabs within strings adding particular underscores
	%	title=\lstname,                                % show the filename of files included with \lstinputlisting; also try caption instead of title
	breaklines=true,                               % sets automatic line breaking
	breakatwhitespace=true,                             % sets if automatic breaks should only happen at whitespace
	frame=single,	                                     % adds a frame around the code
	rulecolor=\color{black},                             % if not set, the frame-color may be changed on line-breaks within not-black text (e.g. comments (green here))
	literate=%
	{Ö}{{\"O}}1
	{Ä}{{\"A}}1
	{Ü}{{\"U}}1
	{ß}{{\ss}}1
	{ü}{{\"u}}1
	{ä}{{\"a}}1
	{ö}{{\"o}}1
	{~}{{\textasciitilde}}1
	% backgroundcolor=\color{white},      % choose the background color; you must add \usepackage{color} or \usepackage{xcolor}; should come as last argument
	% captionpos=b,                       % sets the caption-position to bottom
	% deletekeywords={...},               % if you want to delete keywords from the given language
	% escapeinside={\%*}{*)},             % if you want to add LaTeX within your code
	% extendedchars=true,                 % lets you use non-ASCII characters; for 8-bits encodings only, does not work with UTF-8
	% keepspaces=true,                    % keeps spaces in text, useful for keeping indentation of code (possibly needs columns=flexible)
	% morekeywords={*,...},               % if you want to add more keywords to the set
	% stepnumber=2,                       % the step between two line-numbers. If it's 1, each line will be numbered
}

\begin{document}
	\title{Git \& Github Tutorial}   
	\author{Daniel Schädler \& Jean-Pierre Hotz} 
	\date{22. October 2018}
	%\logo{\includegraphics[scale=0.14]{logo-SF}}
	
	\subtitle{} %Untertitel
	\institute[DHBW KA]{DHBW Karlsruhe}%Angabe des Institutes
	%\logo{\pgfimage[width=2cm,height=2cm]{bnslogo}}%Die Datei hulogo.pdf (bzw. hulogo.png, hulogo.jpg, hulogo.mps bei Verwendung von pdftex als Backend) als Logo auf allen Folien, hier mithilfe des Paketes pgf.
	%\titlegraphic{\includegraphics[width=2cm,height=2cm]{bnslogo}}%Die Datei hulogo.pdf (bzw. analog wie bei \logo auch entsprechendes Format) als Logo nur auf der Titelseite unter Verwendung des Paketes graphicx.
	\subject{Software-Engineering}%Setzt Thema in den PDF-Dokument-Eigenschaften
	\keywords{}%Setzt Schlüsselwörter in den PDF-Dokument-Eigenschaften
	
	\setbeamertemplate{navigation symbols}[vertical]
	
	\begin{frame}
		\titlepage
	\end{frame}
	
	\begin{frame}
		\tableofcontents
	\end{frame}

	\section{General}
	\begin{frame}
		\frametitle{General}\pause
		\begin{itemize}
			\item Git is a \textbf{V}ersion \textbf{C}ontrol \textbf{S}ystem \pause
			\item created in 2005 by Linus Torvalds
		\end{itemize}
	\end{frame}

	\section{Terminology}
	\begin{frame}
		\frametitle{Terminology}\pause
		\begin{itemize}
			\item Commit (History) \pause
			\item (Remote) Repository
		\end{itemize}
	\end{frame}

	\section{Git areas}
	\begin{frame}
		\frametitle{Git areas}
		\begin{figure}[h]
			\centering
			\includegraphics[width=.9\linewidth]{git-areas.png}
			\label{fig1}
		\end{figure}
	\end{frame}

	\section{Use your Github user in Git}
	\begin{frame}
		\frametitle{Use your Github user in Git}\pause
		\begin{itemize}
			\item set your username and e-mail \newline
			\lstinline|$ git config --global user.name "<username>"|\newline
			\lstinline|$ git config --global user.email <e-mail>|\newline
		\end{itemize}
	\end{frame}

	\section{Ignoring files}
	\begin{frame}
		\frametitle{Ignoring files}\pause
		\begin{itemize}
			\item exclude files with \lstinline|.gitignore|-file \pause
			\item contains patterns \pause
			\item \lstinline|*| matches anything \pause
			\item postceding \lstinline|/| matches a folder \pause
			\item preceding \lstinline|/| matches only in the root folder \pause
			\item preceding \lstinline|!| negates the pattern (i.e. includes files)
		\end{itemize}
	\end{frame}

	\begin{frame}[fragile]
		\centering What do the following rules exclude?
		\begin{lstlisting}
/src/
!/src/working/
		\end{lstlisting}
	\end{frame}

	\section{Local repositories}
	\begin{frame}
		\frametitle{Local repositories}\pause
		\begin{itemize}
			\item create new local repository\newline
			\lstinline|$ git init| \pause
			\item clone a remote repository\newline
			\lstinline|$ git clone <URL>|
		\end{itemize}
	\end{frame}

	\section{Staging and committing}
	\begin{frame}
		\frametitle{Staging and committing}\pause
		\begin{itemize}
			\item stage changes\newline
			\lstinline|$ git add (--all / <files>)| \pause
			\item commit all the staged changes\newline
			\lstinline|$ git commit -m "<Message>"| \pause
			\item Commit messages are important!
		\end{itemize}
	\end{frame}

	\section{Display changes and commits}
	\begin{frame}
		\frametitle{Display changes and commits}\pause
		\begin{itemize}
			\item see changes in code\newline
			\lstinline|$ git diff <commit1> <commit2> <files>| \pause
			\item see staged changes\newline
			\lstinline|$ git status| \pause
			\item see commit history\newline
			\lstinline|$ git log|
		\end{itemize}
	\end{frame}

	\section{Remote repositories}
	\begin{frame}
		\frametitle{Remote repositories}\pause
		\begin{itemize}
			\item add remote repository\newline
			\lstinline|$ git remote add <Name> <URL>| \pause
			\item not needed when repository was cloned \pause
			\item set remote repository as up-stream\newline
			\lstinline|$ git branch -u <Name> <Branch>|\newline
			\lstinline|$ git push -u <Name> <Branch>| \pause
			\item push all commits to remote repository\newline
			\lstinline|$ git push| \pause
			\item pulls all commits from remote repository\newline
			\lstinline|$ git pull| 
		\end{itemize}
	\end{frame}

	\begin{frame}
		\frametitle{Remote repositories}\pause
		\begin{itemize}
			\item list tags\newline
			\lstinline|$ git tag --list| \pause
			\item tag a commit\newline
			\lstinline|$ git tag -a <version> -m "<message>" <commit>| \pause
			\item push all tags\newline
			\lstinline|$ git push origin --tags|
		\end{itemize}
	\end{frame}
	
	\section{Branches}
	\begin{frame}
		\frametitle{Branches}\pause
		\includegraphics[width=.725\linewidth]{gitbranches.png}\pause
		\begin{itemize}
			\item different directions the project is heading of into\pause
			\item different features (feature branches)\pause
			\item can be merged again\newline
		\end{itemize}
	\end{frame}
	
	\section{merging}
	\begin{frame}
		\frametitle{merging}\pause
		\begin{itemize}
			\item bringing two branches back together\newline
			\lstinline|$ git merge <branchname>| \pause\newline
			\item merges \lstinline|<branchname>| into current branch\pause
		\end{itemize}
		\includegraphics[width=.725\linewidth]{gitbranchesslim.png}
	\end{frame}
	
	\section{conflict resolving}
	\begin{frame}[fragile]
		\frametitle{conflict resolving}\pause
		\begin{itemize}
			\item conflicts happen when two people edit the same line\pause
			\item[] 
			\begin{lstlisting}
To help you with git,
<<<<<<< HEAD
this line has many infos
=======
I write many different tutorials
>>>>>>> conflicting branch
			\end{lstlisting}
			\item resolve them manually by editing the files\pause\newline
			\item \lstinline|git add <conflicting-file>|
		\end{itemize}
	\end{frame}
	
	\section{rebasing}
	\begin{frame}
		\frametitle{rebasing}\pause
		\begin{itemize}
			\item reapply commits on top of another base tip\pause
			\item keeps the git history straight\pause
			\item allows for \lstinline|--ff-only| merge\newline
			\lstinline|$ git merge --ff-only <branchname>| \pause
		\end{itemize}
		\includegraphics[width=.725\linewidth]{gitrebase.png}
	\end{frame}
	
	\section{Rewriting history}
	\begin{frame}
		\frametitle{Rewriting history}\pause
		\begin{itemize}
			\item You can rewrite git history\pause\newline
			\item \lstinline|$ git rebase -i HEAD~<number>|\newline
			\lstinline|<number>| is the number of commits you want to go back and edit\pause
		\end{itemize}
	\end{frame}
	
	\begin{frame}[fragile]
		\frametitle{Rewriting history}\pause
		\lstinline|$ git rebase -i HEAD~<number>|\newline
		\begin{lstlisting}[basicstyle=\tiny]
pick ce28e71 change gitignore intellij
pick 4067bbb Add chapter about tags
pick 78580c3 Change some slides
pick 1641359 Complete my slides
pick 091611d add current pdf version
pick db3e8b4 add Daniels part to slides

# Rebase 3a4a271..db3e8b4 onto 3a4a271 (7 commands)
#
# Commands:
# p, pick <commit> = use commit
# r, reword <commit> = use commit, but edit the commit message
# e, edit <commit> = use commit, but stop to amend
# s, squash <commit> = use commit, but meld into previous commit
# These lines can be re-ordered; they are executed from top to bottom.                                                           
#
# If you remove a commit here THAT COMMIT WILL BE LOST.
		\end{lstlisting}
	\end{frame}
	
	\section{GitHub (\& others)}
	\begin{frame}
		\frametitle{GitHub (\& others)}\pause
		
		\begin{minipage}{0.4\textwidth} 
				complete copy \newline\newline
				branch protection \newline\newline
				pull requests \newline\newline
				issues \newline

		% \caption{Der Text}
		% \label{Text}
		\end{minipage}
		% Auffüllen des Zwischenraums
		\hfill
		% minipage mit Grafik
		\begin{minipage}{0.4\textwidth}
		
		\includegraphics[width=.725\linewidth]{githublogo.png}\newline

		
		
		\includegraphics[width=.725\linewidth]{bitbucketlogo.png}\newline

		
		\label{Bild} 
		\end{minipage}
		
		
		
	
	\end{frame}
	

\end{document}